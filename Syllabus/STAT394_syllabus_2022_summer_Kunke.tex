\documentclass{article}
\usepackage[margin=0.5in]{geometry}
\usepackage[utf8]{inputenc}
% \usepackage[colorlinks=true]{hyperref}
\usepackage{hyperref}
\usepackage[svgnames]{xcolor}
\definecolor{link}{rgb}{0, 0.5, 0.8}
\hypersetup{
    colorlinks = true,
    allcolors = link
}
% \urlstyle{same}

\title{Math/Stat 394 A: Probability I (Summer 2022)}
\author{Jess Kunke}
\date{June 21 - July 20, 2022}

\begin{document}

\maketitle

\large Welcome!  I look forward to working with you, and I hope you will find this course useful, interesting, and rewarding.

\normalsize

\tableofcontents

\section{Logistics}

\begin{description}
    \item[{Instructor}] Jess Kunke, jkunke@uw.edu
    \item[{Grader}] Shi Feng
    \item[{Prereqs}] Overall, a firm mastery of multivariable calculus and basic combinatorics (counting methods) will serve you well.
    \begin{itemize}
        \item Required: Univariate calculus (MATH 126/136), sums, power series, basic combinations and permutations
        \item Recommended: Any course with combinatorics or discrete math in it (e.g. Math 300 or 381)
    \end{itemize}
    \item[{Lectures}] MWF 8:30 am - 10:40 am. This class is intended to be primarily in person, but I will do my best to provide an effective virtual option so that students may attend virtually as needed and lectures can be recorded.
    \begin{itemize}
        \item May attend in person (\href{https://www.washington.edu/maps/#!/CMU}{\textbf{CMU}} 230) or via Zoom (see Canvas Zoom page for links)
        \item \textbf{Note: Exams are in person only}
        \item We will start promptly at 8:30 due to the accelerated format
        \item Lectures will be recorded; recordings will be available under the Cloud Recordings tab of the Zoom page on Canvas
        \item 1-hour blocks with 10-min break
        \item Mix of lecture and short practice exercises
    \end{itemize}
    \item[{Office Hours}] Two hours/week, time TBD based on student preference
    \item[{Textbook}] Anderson, Seppäläinen, and
    Valkó. \emph{Introduction to Probability.} (available in print or online; you can \href{https://collegestudenttextbook.org/product/introduction-to-probability-ebook/}{\textbf{purchase a digital copy for \$8}})
    \item[{Grade breakdown}] Homework 30\%, Midterm 35\%, Final 35\%
    \item[{Questions}] For conceptual/content questions outside of class, you are encouraged to attend office hours or use Ed Discussion (linked through Canvas). For adminimistrative issues or private concerns, please email the instructor at jkunke@uw.edu.
    \item[{Anonymous Feedback}] Welcome throughout the course; \href{https://forms.gle/VMv2eQ1z7qkqdHJ69}{\textbf{submit here}} or find on Canvas
\end{description}

\section{Course description}

This is the first quarter of a sequence in probability theory. This quarter, we will present the axioms of probability, that is, the tools to measure the uncertainty of some events. We will introduce the concepts of conditional and independent events. Then we will define what are random variables, how do we characterize discrete and continuous random variables by means of their probability mass or density functions. We will define what is the expectation and variance of a random variable, as well as how we compute these quantities using the probability mass or density functions. We will introduce classical distributions, such as the Bernoulli, binomial, geometric, Poisson, uniform, exponential, and Gaussian distributions. Finally, we will introduce the Law of Large Numbers and the Central Limit Theorem, which are the essential theorems that allow us to move from our theory to concrete estimations of some random variables.

\section{Learning objectives}

By the end of MATH/STAT 394, students will be able to...

\begin{enumerate}
    \item Better understand different views of ``randomness" in the context of probability and statistics
    \item Define and apply the following terms/concepts among others: probability, statistics, sample space, probability model, random variable, probability distribution, conditional probability, conditional independence
    \item Identify which probability rules to apply when, and correctly apply them
    \item Take a description of a sampling problem and formulate it in terms of a probability model
    \item Recognize some common probability distributions
    \item Determine when the binomial distribution can be well approximated by normal or Poisson
    \item Apply Bayes' theorem
    \item Calculate expectation and variance from a pdf/pmf
    \item Compute a confidence interval for a sample proportion
\end{enumerate}

\section{Homework}

\textbf{See Canvas for the updated schedule.} There will be 4 problem sets due Mondays at midnight \textbf{except for} July 4th; instead, HW2 is due the previous Friday. The main purpose of the homework assignments is to give you regular hands-on practice with the material.

\begin{description}
    \item[{Submission}] Submit as a \textbf{pdf} via Gradescope (linked from Canvas)
    \begin{itemize}
        \item No late submissions accepted except in extreme circumstances. If you have concerns about completing the work, let me (the instructor) know as soon as possible.
        \item For Gradescope to work, your solution to a given problem can span more than one page, but no page should contain work on more than one problem. e.g. pages 1-2 can be Problem 1, page 3 can be Problem 2, etc, but even if Problem 2 only takes you a quarter of the page, start Problem 3 on a fresh page.
    \end{itemize}
    \item[{Format}] LaTeX is encouraged but not required; see Canvas for a template. It may be a valuable chance to learn LaTeX or improve your LaTeX fluency, and you can always try it just for some and not all of the homeworks. Each homework assignment is written in LaTeX, so if you would like to write up your homework in LaTeX, you can download the sourcecode plus a couple supplementary files from the website, then write in your solutions to each problem.
    \item[{Grading}] \emph{Clear, detailed, and legible} mathematical explanation
    is required to receive full credit. Individual and/or general feedback will be provided. Written regrade requests for a given homework must be submitted via email within three days of the date that grades for that assignment were released.
    \item[{Collaboration}] You are encouraged to work with your classmates, but you must write up your solutions \emph{on your own}. Additionally, I encourage you to struggle with the problems yourself first before hearing other people's ideas because I think you will learn the most this way. If you collaborate with one or more classmates on a problem set, please mark their names at the top of your pdf submission.
\end{description}


\section{Tips for success}

This is an accelerated course, so there is even less time than usual to digest the material.

\begin{itemize}
    \item \textbf{Ask your questions along the way}; I encourage you to ask both me and each other!
    \item \textbf{Start the homework as soon as possible} and pace out your work so that you will have enough time to ask questions and check your work.  I will do my best to respond to questions throughout the course, but I cannot guarantee that I will see or respond to last-minute questions.
    \item \textbf{Do as much quality practice as you can.} The homework is intended to give you useful practice, and try to do some additional practice from the book as well based on what you realize you need to focus on, such as problems similar to ones you missed on the homework/exams, or concepts you feel haven't ``clicked" yet.
    \item Make sure to \textbf{devote some time to struggle with problems yourself before asking others} about them. It may feel inefficient or like a waste of time, but the right kind of struggle is actually the best way to learn. Of course, if you have clarification questions, please ask, and if after wrangling the problem for a while you want to chat with me or with classmates, that's a good time to reach out. However, struggling through how to frame a problem and what steps to take next will teach you a lot more, even if in the end you ask for help, than just clarifying those points from the start. Our focus is more on conceptual understanding, learning when to use what tools, and less on computational skills, so the majority of the problem is figuring out what type of problem you're dealing with and what tools to use in what order.
    \item A great practice tip to make sure you understand something is to \textbf{work through examples that are well-defined and small enough in scope} that you can enumerate all the possibilities. That way you can double-check your answer multiple ways. This tip will serve you well not just in our class but throughout your math studies.
    \item \textbf{Do the reading before each lecture}. You don't have to read every line, and you can set a timer for whatever is a reasonable amount of time for you that day, but whatever you can do to engage with the material before you see it in lecture will likely help you jump in and process it better. Also, it will serve you well in future studies/research to practice using and learning from a textbook.
    \item Above all, \textbf{learn about how you learn best} and do that!
\end{itemize}


\section{Calendar}
\label{sec:org5fd0cdc}
Here is a tentative calendar; see Canvas for the updated schedule.

\begin{center}
\begin{tabular}{rlllll}
Week & Date & Chapter & Topic & Due & Note\\
\hline
1 & 06-22 Wed & App. B-D, & combinatorial analysis; &  & \\
 & & \S 1.1-1.2 & sample space; axioms of probability &  & \\
 & 06-24 Fri & \S 1.3-1.5, 2.1 & random variables; infinitely many trials; conditional prob. &  & \\
\hline
2 & 06-27 Mon & \S 2.2-2.3 & Bayes' formula; independence & HW 1 & \\
 & 06-29 Wed & \S 2.4-2.5 & independent trials; binomial and geometric distributions &  & \\
 & 07-01 Fri & \S 3.1-3.3 & pmf, pdf and cdf; uniform distribution; expectation & HW 2 & \\
\hline
3 & \textit{07-04 Mon} &  & \textit{No Class} &  & \textit{Holiday}\\
 & 07-06 Wed & \S 3.4-3.5 & variance; normal/Gaussian distribution &  & \\
 & 07-08 Fri &  & Midterm &  & Midterm\\
\hline
4 & 07-11 Mon & \S 4.1-4.4 & normal and Poisson approx.; confidence intervals & HW 3 & \\
 & 07-13 Wed & \S4.5 & exponential and geometric distributions &  & \\
 & 07-15 Fri & \S5.2, 6.1 & distribution of a function of a random variable &  & \\
\hline
5 & 07-18 Mon & \S9.1-9.2 & tail bounds; weak law of large numbers; & HW 4 & \\
 &  &  & max/min of iid random variables &  & \\
 & 07-20 Wed &  & Final &  & Final\\
\hline
\end{tabular}
\end{center}








\newpage

\section{Expectations and Conduct}
\label{sec:orge26f9c3}
\begin{description}
\item[{DRS}] If you have accomodations from Disability Resources for Students (DRS), please let the instructor know as soon as possible. DRS can help accommodate conditions including, but not limited to: mental health, attention-related, learning, vision, hearing, physical, or other conditions. DRS can be contacted at 206-543-8924, uwdrs@uw.edu, or \url{http://depts.washington.edu/uwdrs/}.

\item[{Diversity}] Diverse backgrounds, embodiments, and experiences are essential to the critical thinking endeavor at the heart of university education. Therefore, I expect you to follow the \href{https://www.washington.edu/cssc/for-students/student-code-of-conduct/}{UW Student Conduct Code} in your interactions with your colleagues and me in this course by respecting the many social and cultural
differences among us, which may include, but are not limited to: age, cultural background, disability, ethnicity, family status, gender identity and presentation, citizenship and immigration status, national origin, race, religious and political beliefs, sex, sexual orientation, socioeconomic status, and veteran status.

\item[{Academic Integrity}] I take academic integrity seriously and I hope you do too. Potential penalties you may face for cheating on homework or exams include receiving a zero for particular problems, receiving a zero for the entire assignment/exam, and being reported to the university for academic misconduct. Just don't do it. It's not worth it.

\begin{itemize}
\item Tests: this should be obvious, but you may not use unauthorized test materials, look at other students' tests, or obtain help from others during a test.

\item Homework: you are encouraged to work with classmates and discuss your respective ideas and approaches. Some answers to homework problems or similar problems will be available (back of the book, previous solutions, online, etc), and you are free to consult and discuss these. All these other sources (classmates and otherwise) can either help or hinder your learning depending on how you use them. The key to not cheating on homework is simple: \emph{don't plagiarize}. Your submitted solutions must be your own and must demonstrate \emph{your personal understanding}. We are grading you on your own understanding, not on other people's.
\end{itemize}
% \item[{This is an accelerated course}] Therefore, attendance is
% essential and encouraged
% \item[{Practice}] Doing lots of homework problems is the best way to
% learn the material.  Read the exercises and solutions if you don’t
% understand the concepts - they should help you.  If you don’t
% understand the material, try doing extra problems from any of the
% reference texts I listed above. Feel free to come to office hours
% with questions on problems besides the HW.
% \item[{Career}] 
\item[{Recomendation Letters}] If you are looking for a letter of recommendation, I encourage you to ask other course instructors you have had that are professors because their letters will carry more weight than mine as a PhD student.
\end{description}

\section{Optional further reading/references}
\label{references}

Some optional additional references:

\begin{itemize}
\item \emph{\href{https://www.probabilitycourse.com/}{Introduction to Probability, Statistics, and Random Processes}} by Hossein Pishro-Nik. Freely available.

\item \emph{A First Course in Probability}, by Sheldon Ross. A classic. Since this would be used for your reference only, the edition does not particularly matter. However, I would suggest avoiding the latest which is very expensive.

\item \emph{\href{https://math.dartmouth.edu/\~prob/prob/prob.pdf}{Grinstead and Snell's Introduction to Probability}}. Freely
available. Emphasizes building intuition and working problems through coding.

\item \emph{\href{http://www.utstat.toronto.edu/mikevans/jeffrosenthal/book.pdf}{Probability and StatisticsThe Science of Uncertainty}} by Evans and Rosenthal. Freely available.

\item \emph{\href{http://mitran-lab.amath.unc.edu/courses/MATH768/biblio/introduction-to-prob-models-11th-edition.PDF}{Introduction to Probability Models}}, 11th Edition, Sheldon Ross. Freely available.
\end{itemize}

\noindent Beyond those reference texts, here are some more pop probability and statistics books that you may find enjoyable:

\begin{itemize}
\item \href{http://www.amazon.com/Chance-Guide-Gambling-Market-Everything/dp/1560257946}{Chance:  A Guide to Gambling, Love, the Stock Market, and Just About Everything Else}, by Amir D. Aczel
\item \href{http://www.amazon.com/How-Lie-Statistics-Darrell-Huff/dp/0393310728/}{How to Lie with Statistics}, by Darrell Huff.
\item \href{http://www.amazon.com/Chances-Are-Probability-Michael-Kaplan/dp/0143038346/}{Chances Are:  Adventures in Probability}, by Michael and Ellen Kaplan.
\item \href{http://www.amazon.com/Drunkards-Walk-Randomness-Rules-Vintage/dp/0307275175}{The Drunkard’s Walk:  How Randomness Rules Our Lives}, by Leonard Mlodinow.
\item \href{http://www.amazon.com/Lady-Tasting-Tea-Statistics-Revolutionized/dp/0805071342}{The Lady Tasting Tea:  How Statistics revolutionized Science in the Twentieth Century}, by David Salsburg.
\item \href{https://www.goodreads.com/en/book/show/4443547-the-unfinished-game}{The Unfinished Game: Pascal, Fermat, and the Invention of Probability}, by Keith Devlin.
\item \href{https://www.goodreads.com/en/book/show/1445847.Randomness}{Randomness}, by Deborah J. Bennett.
\item \href{https://www.goodreads.com/book/show/38315.Fooled\_by\_Randomness?from\_search=true\&from\_srp=true\&qid=Djljw9C0Ni\&rank=1}{Fooled by Randomness: The Hidden Role of Chance in the Markets and in Life}, by Nassim Nicholas Taleb.
\end{itemize}


% \noindent These are just suggestions and a couple of the above books
%   are more about statistics than probability. I list these here for
%   the student interested in expanding their understanding and
%   enjoyment of the domain above and beyond what may be covered in
%   this or any other course. There are books about probability and
%   history, probability and logic, probability and philosophy, and
%   many more. We will each have our own preferences for probability
%   pleasure reading, and I suggest you take the time to find a book
%   that may make probability a little more interesting and real for
%   you, in a context very different from your course work.
\end{document}
